\documentclass[11pt]{article}
\usepackage[a4paper,margin=1in]{geometry}
\usepackage[T1]{fontenc}
\usepackage{lmodern}
\usepackage{xcolor}
\usepackage{hyperref}
\usepackage{enumitem}
\usepackage{minted}

\hypersetup{
    colorlinks=true,
    linkcolor=blue!70!black,
    urlcolor=blue!70!black,
    citecolor=blue!70!black
}

% Minted configuration for syntax highlighting
\usemintedstyle{tango}
\setminted{
    fontsize=\small,
    frame=single,
    framesep=2mm,
    baselinestretch=1.1,
    linenos=true,
    numbersep=5pt,
    breaklines=true,
    breakanywhere=true,
    tabsize=2
}

% Custom colors for code background
\definecolor{codebg}{RGB}{248,248,248}
\setminted{bgcolor=codebg}

\title{React-Essentials (Summary)}
\author{}
\date{2026-01-19}

\begin{document}
\maketitle
\subsection{Key Takeaways}
\begin{itemize}[leftmargin=*]
\item A component can be defined like any other JS function. Its return body must be some sort of mark-up. Furthermore, components must be capitalized
\item Components (\texttt{.jsx}) wrap javascript, html, and css together into one file.
\item Since .jsx code is built with the React compiler, the code that is written is not the same code that is delivered to the end-user.
\item Essentially, React compiles the Component Tree into the DOM Tree
\item To insert a component into the main program body i.e: App() => \{\}.
\end{itemize}
\section{React Essentials}
\subsubsection{Components}
Components (\texttt{.jsx}) wrap javascript, html, and css together into one file. Helping keep front-end codebases relatively small.

\begin{itemize}[leftmargin=*]
\item A component can be defined like any other JS function. Its return body must be some sort of mark-up. Furthermore, components must be capitalized
\end{itemize}
To insert a component into the main program body i.e: App() => \{\}. We can either insert the component like any other HTML tag, or you can have the component be self-closing. For example:

\begin{minted}{jsx}
function Body(){
    return (
        <p> Hi! </p>
    );
}

function App(){
    return (
        <title> Website </title>
        <Body/> //or <Body> </Body>
    );
}

export default App
\end{minted}

\subsubsection{Component Tree}
Since .jsx code is built with the React compiler, the code that is written is not the same code that is delivered to the end-user. Using \texttt{ReactDOM}, we are able to serve a .jsx file as an entry-point into a specific element. In the starting-project example, we have a index.html file that contains a component (id = "root") which we use as the entry-point for the \texttt{App.jsx} file with following ReactDOM methods in our index.jsx file:

\begin{minted}{jsx}
import App from "./App.jsx";
import ReactDOM from "react-dom/client";

const entryPoint = document.getElementById("root");
ReactDOM.createRoot(entryPoint).render(<App />);
\end{minted}

Essentially, React compiles the Component Tree into the DOM Tree

\end{document}
